% Compiled and edited by Brian Bi
\documentclass[9pt]{extarticle}
\setlength{\parindent}{0.0in}
\usepackage{amsmath}
\usepackage{multicol}
\usepackage{hyperref}
\usepackage[landscape,a4paper,twoside=false,top=15mm,bottom=10mm,left=10mm,right=10mm]{geometry}
\usepackage{fancyhdr}
\pagestyle{fancy}
\setlength{\headsep}{0mm}
\setlength{\voffset}{8mm}
\fancyhead[EOL]{Moradonellani (Politecnico di Milano)}
\fancyhead[EOR]{\thepage}
\cfoot{}
\usepackage{listings}
\usepackage{titlesec}
\usepackage{color}
\lstset{
	tabsize=2,
    basicstyle=\ttfamily\scriptsize,
    %upquote=true,
    aboveskip={1.5\baselineskip},
    columns=fixed,
    showstringspaces=false,
    extendedchars=true,
    breaklines=true,
    prebreak = \raisebox{0ex}[0ex][0ex]{\ensuremath{\hookleftarrow}},
    frame=single,
    rulecolor=\color[rgb]{0.75,0.75,0.75},
    showtabs=false,
    showspaces=false,
    showstringspaces=false,
    keywordstyle=\color[rgb]{0,0,1},
    commentstyle=\color[rgb]{0.133,0.545,0.133},
    stringstyle=\color[rgb]{0.627,0.126,0.941},
}
\titlespacing*{\section} {0pt}{0.5ex plus 1ex minus .2ex}{0.3ex plus .2ex}
\titlespacing*{\subsection} {0pt}{0.25ex plus 1ex minus .2ex}{0.3ex plus .2ex}

\begin{document}
% Disable balancing of columns on last page, which is ugly
\begin{multicols*}{3}
% Want compact table of contents, but normal spacing between paragraphs later on
\setlength{\parskip}{0.0in}
\tableofcontents
\setlength{\parskip}{0.1in}
% New codebook
\section{Flow and matching}

\subsection{Hungarian algorithm} % https://raw.githubusercontent.com/ADJA/algos/master/Graphs/HungarianMatching.cpp
\lstinputlisting[language=c++]{hungarian.cpp}

\subsection{Max bipartite matching} % Stanford
\lstinputlisting[language=c++]{bipartite.cpp}

\pagebreak

\subsection{Min-cost max-flow (successive shortest paths)}
% Frank Chu and Igor Naverniouk (modified by Brian Bi)
\lstinputlisting[language=c++]{mcmf4.cpp}

\subsection{Max flow (Dini\'c)} % Stanford
\lstinputlisting[language=c++]{dinic.cpp}

\subsection{Min-cost matching} % Stanford
\lstinputlisting[language=c++]{bipartite-mincost.cpp}

\subsection{Global min cut (Stoer--Wagner)} % Stanford
\lstinputlisting[language=c++]{mincut.cpp}

\subsection{K\"onig's Theorem (Text)} % Jacob Plachta and Brian Bi
In any bipartite graph, the number of edges in a maximum matching equals the
number of vertices in a minimum vertex cover. To exhibit the vertex cover:
\begin{enumerate}
\item Find a maximum matching
\item Change each edge \textbf{used} in the matching into a directed edge from
\textbf{right to left}
\item Change each edge \textbf{not used} in the matching into a directed edge
from \textbf{left to right}
\item Compute the set $T$ of all vertices reachable from unmatched vertices on
the left (including themselves)
\item The vertex cover consists of all vertices on the right that are
\textbf{in} $T$, and all vertices on the left that are \textbf{not in} $T$
\end{enumerate}

\pagebreak

\subsection{General Unweighted Maximum Matching (Edmonds' algorithm)}
% Ilya Grebnov (modified by Brian Bi)
\lstinputlisting[language=c++]{general-matching.cpp}

\subsection{Minimum Edge Cover (Text)} % Brian Bi
If a minimum edge cover contains $C$ edges, and a maximum matching contains $M$
edges, then $C + M = |V|$. To obtain the edge cover, start with a maximum
matching, and then, for every vertex not matched, just select some edge
incident upon it and add it to the edge set. 

\subsection{Stable Marriage Problem (Gale--Shapley algorithm)} % Brian Bi
\lstinputlisting[language=c++]{stablemp.cpp}

\section{Geometry}

\subsection{Miscellaneous Geometry} % Stanford
\lstinputlisting[language=c++]{geom-2d.cpp}

\subsection{3D Geometry}
% Stanford (translated, modified, and expanded by Wesley May and Brian Bi)
\lstinputlisting[language=c++]{geom-3d.cpp}

\subsection{Minimum Enclosing Disk (Welzl's Algorithm)} % Brian Bi
\lstinputlisting[language=c++]{welzl.cpp}

\subsection{Convex hull} % Brian Bi
\lstinputlisting[language=c++]{monotone.cpp}

% TODO: Add faster one
\subsection{Slow Delaunay triangulation} % Stanford
\lstinputlisting[language=c++]{delaunay.cpp}

\subsection{Fast Delaunay triangulation}
\lstinputlisting[language=c++]{fast-delaunay.cpp}

\subsection{Pick's Theorem (Text)} % Wesley May
For a polygon with all vertices on lattice points, $A = i + b/2 - 1$, where $A$
is the area, $i$ is the number of lattice points strictly within the polygon,
and $b$ is the number of lattice points on the boundary of the polygon. (Note,
there is no generalization to higher dimensions)

\section{Math Algorithms}

\subsection{Gaussian elimination for square matrices of full rank; finds
inverses and determinants} % Stanford
\lstinputlisting[language=c++]{gaussian.cpp}

\subsection{Sieve of Eratosthenes} % Jimmy Mårdell (Yarin)
\lstinputlisting[language=c++]{yarin.cpp}

\subsection{Modular arithmetic and linear Diophantine solver} % Stanford
\lstinputlisting[language=c++]{modular.cpp}

\subsection{Reduced row echelon form (RREF), matrix rank} % Stanford
\lstinputlisting[language=c++]{rref.cpp}

\subsection{Solving linear systems (Text)} % Brian Bi
To solve a general system of linear equations, put it into matrix form and
compute the reduced row echelon form. For example,
\begin{align*}2x + y &= 5 \\ 3x + 2y &= 6\end{align*}
corresponds to the matrix
\[ \left[ \begin{array}{cc|c} 2 & 1 & 5 \\ 3 & 2 & 6 \end{array} \right] \]
with RREF
\[ \left[ \begin{array}{cc|c} 1 & 0 & 4 \\ 0 & 1 & -3 \end{array} \right] \]
After row reduction, if any row has a 1 in the rightmost column and 0
everywhere else, then the system is inconsistent and has no solution.
Otherwise, to find a solution, set the variable corresponding to the leftmost 1
in each column equal to the corresponding value in the rightmost column, and
set all other variables to 0. Ignore rows consisting entirely of 0. The
solution is unique iff the rank of the matrix equals the number of variables.

\subsection{Fast Fourier transform (FFT)} % Stanford
\lstinputlisting[language=c++]{fft.cpp}

\subsection{Simplex algorithm} % Stanford
\lstinputlisting[language=c++]{simplex.cpp}

\subsection{Number Theoretic Transform (NTT)}
\lstinputlisting[language=c++]{ntt.cpp}

\subsection{Fast factorization (Pollard rho) and primality testing
(Rabin--Miller)} % Qiyu Zhu
\lstinputlisting[language=c++]{pollard-rho.cpp}

\subsection{Euler's Totient} % Andre Hahn Pereira
\lstinputlisting[language=c++]{totient.cpp}

\section{Graphs}

\subsection{Strongly connected components} % Stanford (modified by Brian Bi)
\lstinputlisting[language=c++]{scc.cpp}

\subsection{Bridges} % Andre Hahn Pereira
\lstinputlisting[language=c++]{bridges.cpp}

\subsection{Eulerian path} % Andre Hahn Pereira
\lstinputlisting[language=c++]{eulerian.cpp}

\subsection{Lowest Common Ancestor (Pseudocode)} % Jacob Plachta
Tarjan's offline algorithm (requires $O(N)$ disjoint set operations).
\begin{lstlisting}
function TarjanOLCA(u)
    MakeSet(u);
    u.ancestor := u;
    for each v in u.children do
        TarjanOLCA(v);
        Union(u,v);
        Find(u).ancestor := u;
    u.colour := black;
    for each v such that {u,v} in P do
        if v.colour == black
            print "Tarjan's Least Common Ancestor of " + u +
                " and " + v + " is " + Find(v).ancestor + ".";
\end{lstlisting}
This function is called on the root of the tree. The set P of pairs of nodes to
query must be specified in advance. Each node is initially white, and is
colored black after it and all its children have been visited. The lowest
common ancestor of the pair {u,v} is available as Find(v).ancestor immediately
(and only immediately) after u is colored black, provided v is already black.
Otherwise, it will be available later as Find(u).ancestor, immediately after v
is colored black.

\section{Data Structures}

\subsection{Suffix arrays} % Stanford
\lstinputlisting[language=c++]{suffix-array.cpp}

\subsection{Binary Indexed Tree (BIT) / Fenwick Tree} % Brian Bi
\lstinputlisting[language=c++]{BIT.cpp}

\subsection{BIT with range query/updates}
\lstinputlisting[language=c++]{BIT-range-range.cpp}

\subsection{BIT with range updates} % Brian Bi
\lstinputlisting[language=c++]{BIT-range.cpp}

\subsection{Segment Tree with Lazy Propagation} % Brian Bi
\lstinputlisting[language=c++]{segtree.cpp}

\subsection{Size-Balanced Tree}
% Tom Yubing Dong (tomtung) (modified by Brian Bi)
\lstinputlisting[language=c++]{SBT.cpp}

\section{Number Theory Reference}
\subsection{Polynomial Coefficients (Text)} % Brian Bi
$(x_1 + x_2 + ... + x_k)^n = \sum_{c_1 + c_2 + ... + c_k = n}
\frac{n!}{c_1! c_2! ... c_k!} x_1^{c_1} x_2^{c_2} ... x_k^{c_k}$

\subsection{M\"obius Function (Text)} % Brian Bi
$\mu(n) = \begin{cases}
0 & \text{$n$ not squarefree} \\
1 & \text{$n$ squarefree w/ even no. of prime factors} \\
1 & \text{$n$ squarefree w/ odd no. of prime factors} \\
\end{cases}$
Note that $\mu(a) \mu(b) = \mu(ab)$ for $a, b$ relatively prime
Also $\sum_{d \mid n} \mu(d) = \begin{cases} 1 & \text{if $n = 1$} \\
0 & \text{otherwise} \end{cases}$

\textbf{M\"obius Inversion}
If $g(n) = \sum_{d|n} f(d)$ for all $n \ge 1$, then
$f(n) = \sum_{d|n} \mu(d)g(n/d)$ for all $n \ge 1$.

\subsection{Burnside's Lemma (Text)} % Wesley May and Brian Bi
The number of orbits of a set $X$ under the group action $G$ equals the average
number of elements of $X$ fixed by the elements of $G$.

Here's an example. Consider a square of $2n$ times $2n$ cells. How many ways
are there to color it into $X$ colors, up to rotations and/or reflections?
Here, the group has only 8 elements (rotations by 0, 90, 180 and 270 degrees,
reflections over two diagonals, over a vertical line and over a horizontal
line). Every coloring stays itself after rotating by 0 degrees, so that
rotation has $X^{4n^2}$ fixed points. Rotation by 180 degrees and reflections
over a horizonal/vertical line split all cells in pairs that must be of the
same color for a coloring to be unaffected by such rotation/reflection, thus
there exist $X^{2n^2}$ such colorings for each of them. Rotations by 90 and 270
degrees split cells in groups of four, thus yielding $X^{n^2}$ fixed colorings.
Reflections over diagonals split cells into $2n$ groups of 1 (the diagonal
itself) and $2n^2-n$ groups of 2 (all remaining cells), thus yielding
$X^{2n^2-n+2n}=X^{2n^2+n}$ unaffected colorings.  So, the answer is
$(X^{4n^2}+3X^{2n^2}+2X^{n^2}+2X^{2n^2+n})/8$.

\section{Miscellaneous}

\subsection{Knuth--Morris--Pratt (KMP)} % Stanford
\lstinputlisting[language=c++]{KMP.cpp}

\subsection{Z-Algorithm}
\lstinputlisting[language=c++]{Z-algorithm.cpp}

\subsection{2-SAT} % Brian Bi
\lstinputlisting[language=c++]{2sat.cpp}

\subsection{Convex hull trick} % Brian Bi
\lstinputlisting[language=c++]{chtrick.cpp}

\subsection{Shunting Yard (Pseudocode)} % Wesley May
\begin{lstlisting}
// Add '(' to start of expression, and ')' to end.
O = empty vector of tokens (values or operators)
S = empty stack of tokens (brackets or operators)
for each token:
  if token == value:
    O.push(token)
  else if token == '(':
    S.push(token)
  else if token == ')':
    while S.top() != '(':
      O.push(S.top())
      S.pop()
    S.pop()
  else:
    // Note: If token is a right-associative operator (^), this should be <=.
	// priority('(') < priority('+') < priority('*').
    while priority(S.top()) < priority(token):
      O.push(S.top())
      S.pop()
    S.push(token)
// Finally, evaluate O as a postfix expression.
\end{lstlisting}

\subsection{Binary search} % Brian Bi
\lstinputlisting[language=c++]{bsearch.c}

\subsection{Bignums (C++, slow)} % Brian Bi
\lstinputlisting[language=c++]{bignum.cpp}

\subsection{All nearest smaller values} % Brian Bi
\lstinputlisting[language=c++]{ansv.cpp}

\subsection{Longest palindromic substring} % Brian Bi
\lstinputlisting[language=c++]{manacher.cpp}

\subsection{Union-Find}
\lstinputlisting[language=c++]{union-find.cpp}

\subsection{Bellman-Ford}
\lstinputlisting[language=c++]{bellman-ford.cpp}

\subsection{Floyd-Washall (All Pair Shortest Path)}
\lstinputlisting[language=c++]{floyd-warshall.cpp}

\subsection{Eulerian Tour}
\lstinputlisting[language=c++]{eulerian-tour.cpp}

\subsection{NTT constants generator}
\lstinputlisting[language=c++]{ntt-gen.cpp}

\noindent\rule{9cm}{0.3pt}

Paolo Battellani \\
Marco Donadoni \\
Edoardo Morassutto

\end{multicols*}
\end{document}
